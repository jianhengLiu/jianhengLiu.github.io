%% start of file `template.tex'.
%% Copyright 2006-2013 Xavier Danaux (xdanaux@gmail.com).
%
% This work may be distributed and/or modified under the
% conditions of the LaTeX Project Public License version 1.3c,
% available at http://www.latex-project.org/lppl/.



\documentclass[11pt,a4paper,sans]{moderncv}        % possible options include font size ('10pt', '11pt' and '12pt'), paper size ('a4paper', 'letterpaper', 'a5paper', 'legalpaper', 'executivepaper' and 'landscape') and font family ('sans' and 'roman')

% modern themes
\moderncvstyle{banking}                            % style options are 'casual' (default), 'classic', 'oldstyle' and 'banking'
\moderncvcolor{blue}                                % color options 'blue' (default), 'orange', 'green', 'red', 'purple', 'grey' and 'black'
%\renewcommand{\familydefault}{\sfdefault}         % to set the default font; use '\sfdefault' for the default sans serif font, '\rmdefault' for the default roman one, or any tex font name
%\nopagenumbers{}                                  % uncomment to suppress automatic page numbering for CVs longer than one page

% character encoding
\usepackage[utf8]{inputenc}                       % if you are not using xelatex ou lualatex, replace by the encoding you are using
% \usepackage{CJKutf8}                              % if you need to use CJK to typeset your resume in Chinese, Japanese or Korean
% Uncomment to enable Chinese; needs XeLaTeX
\usepackage{ctex}

% adjust the page margins
\usepackage[scale=0.75]{geometry}
%\setlength{\hintscolumnwidth}{3cm}                % if you want to change the width of the column with the dates
%\setlength{\makecvtitlenamewidth}{10cm}           % for the 'classic' style, if you want to force the width allocated to your name and avoid line breaks. be careful though, the length is normally calculated to avoid any overlap with your personal info; use this at your own typographical risks...

\usepackage{import}

% personal data
\name{Jianheng}{Liu}
%\title{Curriculum Vitae}                               % optional, remove / comment the line if not wanted
%\address{Sanchez de Bustamante, 731, Buenos Aires, Argentina}{}{}% optional, remove / comment the line if not wanted; the "postcode city" and and "country" arguments can be omitted or provided empty
\phone[mobile]{+86 15625293598}                   % optional, remove / comment the line if not wanted
%\phone[fixed]{01234 123456}                    % optional, remove / comment the line if not wanted
%\phone[fax]{+3~(456)~789~012}                      % optional, remove / comment the line if not wanted
\email{liujianhengchris@qq.com}                               % optional, remove / comment the line if not wanted
\homepage{https://github.com/jianhengLiu}                         % optional, remove / comment the line if not wanted  
\extrainfo{\href{https://jianhengliu.github.io/}{https://jianhengliu.github.io/}}                 % optional, remove / comment the line if not wanted
% \photo[64pt][0.4pt]{potrait.jpeg}                       % optional, remove / comment the line if not wanted; '64pt' is the height the picture must be resized to, 0.4pt is the thickness of the frame around it (put it to 0pt for no frame) and 'picture' is the name of the picture file
%\quote{Some quote}                                 % optional, remove / comment the line if not wanted

% to show numerical labels in the bibliography (default is to show no labels); only useful if you make citations in your resume
%\makeatletter
%\renewcommand*{\bibliographyitemlabel}{\@biblabel{\arabic{enumiv}}}
%\makeatother
%\renewcommand*{\bibliographyitemlabel}{[\arabic{enumiv}]}% CONSIDER REPLACING THE ABOVE BY THIS

% bibliography with mutiple entries
%\usepackage{multibib}
%\newcites{book,misc}{{Books},{Others}}
%----------------------------------------------------------------------------------
%            content
%----------------------------------------------------------------------------------
\begin{document}
%\begin{CJK*}{UTF8}{gbsn}                          % to typeset your resume in Chinese using CJK
%-----       resume       ---------------------------------------------------------

\makecvtitle

\small{I am currently a postgraduate in Harbin Institute of Technology (Shenzhen), China, supervised by \textbf{Prof. Haoyao Chen}. I obtained my bachelor degree at Harbin Institute of Technology (Shenzhen), China in 2021.

My research interest lies at \textbf{Robotics and Autonomous Systems, Localization and Mapping, Motion Planning} and \textbf{NeRF}. 

Check out more information and multimedia of research experiences at \href{https://jianhengliu.github.io}{https://jianhengliu.github.io}.

\section{Education}

\vspace{5pt}

\begin{itemize}

\item{\cventry{2021/09--Present}{Control Science and Engineering (Master degree)}{Harbin Institute of Technology (Shenzhnen)}{Recommended exemption Graduate}{\textit{}}{}}

\item{\cventry{2017/09--2021/06}{Automation (Bachelor degree)}{Harbin Institute of Technology (Shenzhnen)}{Rank: 15/70}{\textit{}}{}}

\end{itemize}

\vspace{2pt}

\section{Publications}

\vspace{5pt}

\begin{itemize}
    

\item{\textbf{RGB-D Inertial Odometry for a Resource-restricted Robot in Dynamic Environments}

\small{\textbf{Jianheng Liu}, XuanFu Li, Yueqian Liu and Haoyao Chen. \textbf{RA-L and IROS, 2022}}
}

\vspace{3pt}

\item{\textbf{Sampling-Based View Planning for MAVs in Active Visual-inertial State Estimation}

\small{Zhengyu Hua, Jiabi Sun, Fengyu Quan, Haoyao Chen, \textbf{Jianheng Liu}, Yunhui Liu. \textbf{IROS, 2022}}
}

\vspace{3pt}

\item{\textbf{Vision-encoder-based Payload State Estimation for Autonomous MAV With a Suspended Payload}

\small{\textbf{Jianheng Liu}*, Yunfan Ren*, Haoyao Chen and Yunhui Liu. \textbf{IROS, 2021}}
}

\footnotesize{* equal contribution}

\end{itemize}

\section{Honor \& Awards}

\vspace{6pt}

\begin{itemize}

\item{Graduate Academic Scholarship of First-class (2021-2022), Undergraduate Academic Scholarship of First-class (2019-2020), Third-class (2018-2019), Second-class (2017-2018)}

\item{National ROBOCON competition of First Price (2020), Second Price (2019)}

\item{the Third Prize for 2019 National Challenge Cup, the Bronze Prize for 2019 Internet plus of Heilongjiang Province, the Golden Price for 2019  ZuGuang Cup of Harbin Institute of Technology (Shenzhen)}

\item{the Second Prize for 2018 National English Competition for College Strudents}

\item{the Grand Prize for the second International Youth Drone Competition}

\end{itemize}


\section{Research Experiences}

\vspace{6pt}

\begin{itemize}

\item \textbf{RGB-D Inertial Odometry for a Resource-restricted Robot in Dynamic Environments:} 

\textbf{Jianheng Liu}, XuanFu Li, Yueqian Liu and Haoyao Chen. \textbf{RA-L and IROS, 2022}

Dynamic-VINS is a real-time RGB-D Visual Inertial Odometry (VIO) system for resource-restricted robots in dynamic environments. It is extended based on \href{https://github.com/HKUST-Aerial-Robotics/VINS-Mono}{VINS-Mono}. It combines object detection and RGB-D cameras for dynamic feature recognition to reduce the computational cost, achieving an effect comparable to semantic segmentation. It adopts grid-based feature detection and proposes a fast and efficient method to extract high-quality FAST feature points. A competitive localization accuracy and robustness in dynamic environments are shown in a real-time application on  resource-restricted platforms.

% \begin{itemize}
% \item \textbf{Referred Code:} \href{https://github.com/HITSZ-NRSL/Dynamic-VINS.git}{Dynamic-VINS}
% \item \textbf{Referred Video:} \href{https://youtu.be/y0U1IVtFBwY}{Youtube}, \href{https://www.bilibili.com/video/BV1bF411t7mx}{Bilibili}
% \end{itemize}
\vspace{6pt}

\item \textbf{VINS-RGBD-FAST:} VINS-RGBD-FAST is a SLAM system based on VINS-RGBD. I do refinements both in frontend and backend to improve the system's efficiency in resource-constrained embedded paltform, like HUAWEI Atlas 200DK, Raspberry Pi. Furthermore, we made this system as a module and applied it into UAV as a state feedback to track a generative trajectory stably. 

% \begin{itemize}
% \item \textbf{Referred Code:} \href{https://github.com/jianhengLiu/VINS-RGBD-FAST}{VINS-RGBD-FAST}.
% \end{itemize}
\vspace{6pt}

\item \textbf{Vision-encoder-based Payload State Estimation for Autonomous MAV With a Suspended Payload:} 

\textbf{Jianheng Liu}, Yunfan Ren, Haoyao Chen and Yunhui Liu. \textbf{IROS, 2021}

A novel real-time system for estimating the payload position; the system consists of a monocular fisheye camera and a encoder-based device. A Gaussian fusion-based estimation algorithm is developed to obtain the payload state estimation. Based on the robust payload position estimation, a payload controller is presented to ensure the reliable tracking performance on aggressive trajectories.

% \begin{itemize}
% \item \textbf{Referred Code:} \href{https://github.com/jianhengLiu/Vision-encoder-based-Payload-State-Estimator}{Vision-encoder-based-Payload-State-Estimator}
% \item \textbf{Referred Video:} \href{https://www.bilibili.com/video/BV1Qq4y1U7n4?share_source=copy_web}{Bilibili}
% \end{itemize}
\vspace{6pt}

\item \textbf{MatRix:} A extreme interesting prototype developed in 2020 XBOT PARK Smart Product Innovation Boot Camp. An interactive smart carpet, which can achieve infinite splicing through the magnetic suction connector with anti-dull design. MatRix can be used as your home intelligent terminal, game console, decoration and so on.

% \begin{itemize}
% \item \textbf{Referred Video:} \href{https://www.bilibili.com/video/BV1gb4y127by?share_source=copy_web}{Bilibili}
% \end{itemize}

\vspace{6pt}

\item \textbf{quad-controller-SE3 \& FlightController:} quadrotor controller based on PX4/mavros and SE3 geometric control. And I also develop a simulation based on CoppeliaSim software to compute the desired thrust and torque of quadroter according to dynamic modelling, and use distribution matrix to decide the motor's speed. Furthermore, I conduct a trajectory tracking controller to follow a generative minimum snap trajectory for experiment.

% \begin{itemize}
% \item \textbf{Referred Code:} \href{https://github.com/jianhengLiu/quad_controller_SE3}{quad-controller-SE3}, \href{https://github.com/jianhengLiu/FlightController}{FlightController}

% \item \textbf{Referred Video:} \href{https://www.bilibili.com/video/BV1xv411w7Md?share_source=copy_web}{Bilibili-1}, \href{https://www.bilibili.com/video/BV1rq4y1N76T?share_source=copy_web}{Bilibili-2}
% \end{itemize}

\vspace{6pt}

\item \textbf{BezierTrajGenerator \& MinimumSnapTrajGenerator \& 
MapManager:} Trajectory Generator based on Bezier Curve and Minimum Snap for autonomous robot. And I develop a 2D Map Manager for the verification and visualization for different algorithms.

% \begin{itemize}
% \item \textbf{Referred Code:} \href{https://github.com/jianhengLiu/BezierTrajGenerator}{BezierTrajGenerator}, \href{https://github.com/jianhengLiu/MinimumSnapTrajGenerator}{MinimumSnapTrajGenerator}, \href{https://github.com/jianhengLiu/MapManager}{MapManager}
% \end{itemize}

\vspace{6pt}

\item \textbf{CoppeliaSim/V-Rep Steeling Wheel Robot Tutorial:} A detailed tutorial for a CoppeliaSim/V-Rep beginner to construct their own Steeling Wheel Robot and control it via ROS.

% \begin{itemize}
% \item \textbf{Referred Code:} \href{https://github.com/jianhengLiu/CoppeliaSim_Steeringwheel_Tutorial}{CoppeliaSim-Steeringwheel-Tutorial}
% \end{itemize}
\vspace{6pt}

\item \textbf{Manipulator-GUI:} C++ Course Project (Complied in CodeBlocks). A three dimentional manipulator's forword/inverse kinematics calculation and visualization.

% \begin{itemize}
% \item \textbf{Referred Code:} \href{https://github.com/jianhengLiu/Manipulator_GUI}{Manipulator-GUI}
% \end{itemize}

\end{itemize}


% Publications from a BibTeX file without multibib
%  for numerical labels: \renewcommand{\bibliographyitemlabel}{\@biblabel{\arabic{enumiv}}}% CONSIDER MERGING WITH PREAMBLE PART
%  to redefine the heading string ("Publications"): \renewcommand{\refname}{Articles}
\nocite{*}
\bibliographystyle{plain}
\bibliography{publications}                        % 'publications' is the name of a BibTeX file

% Publications from a BibTeX file using the multibib package
%\section{Publications}
%\nocitebook{book1,book2}
%\bibliographystylebook{plain}
%\bibliographybook{publications}                   % 'publications' is the name of a BibTeX file
%\nocitemisc{misc1,misc2,misc3}
%\bibliographystylemisc{plain}
%\bibliographymisc{publications}                   % 'publications' is the name of a BibTeX file

%-----       letter       ---------------------------------------------------------

\end{document}


%% end of file `template.tex'.
